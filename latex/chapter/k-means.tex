% subsection K-Means

\subsection{K-Means}\label{sec:k-means}

For the first run k-means was applied directly to the untouched stock prizes. That means the distance between two stocks is defined by

\begin{equation}\label{eq:dist-direct}
d_{i,j} = \sum^{n}_{k = 1} {(p_{i,k} - p_{j,k})^2},
\end{equation}

where $p_{i,k}$ is the closing prize of stock $i$ on day $k$ and $n$ the total number of days in the dataset. In simpler terms, the distance of two stocks is defined by the squared area in between both charts.

% k-means, euclidean direct on stock prize

\begin{table}
	\centering
	\caption{Crosstab k-means clustering directly on the stock prize, ($\chi^2 = 113.58$, $p = 0.1667$).}
	\label{tbl:k-means-euc-direct}
	\begin{tabular}{c rrrrrrrrrrr} \toprule
		 & \multicolumn{11}{c}{k-means cluster center}     \\ \cmidrule{2-12}
		GICS & 0 & 1  & 2 & 3  & 4  & 5 & 6  & 7 & 8 & 9  & 10 \\ \midrule
		10 & 11 & 0 & 3  & 0 & 0 & 0 & 11 & 1  & 0 & 0 & 6  \\
		15 & 5  & 0 & 8  & 0 & 1 & 1 & 5  & 1  & 0 & 0 & 4  \\
		20 & 6  & 0 & 11 & 0 & 7 & 0 & 22 & 7  & 0 & 3 & 11 \\
		25 & 24 & 1 & 7  & 1 & 3 & 2 & 24 & 5  & 1 & 0 & 15 \\
		30 & 4  & 0 & 6  & 0 & 0 & 0 & 12 & 2  & 0 & 0 & 10 \\
		35 & 5  & 0 & 9  & 0 & 4 & 1 & 12 & 11 & 1 & 2 & 14 \\
		40 & 14 & 0 & 7  & 0 & 2 & 1 & 20 & 6  & 0 & 0 & 17 \\
		45 & 17 & 2 & 11 & 0 & 2 & 0 & 11 & 7  & 0 & 0 & 19 \\
		50 & 2  & 0 & 0  & 0 & 0 & 0 & 1  & 0  & 0 & 0 & 0  \\
		55 & 8  & 0 & 3  & 0 & 0 & 0 & 9  & 0  & 0 & 0 & 8  \\
		60 & 9  & 0 & 8  & 0 & 2 & 1 & 7  & 2  & 0 & 0 & 4 \\ \bottomrule
	\end{tabular}
\end{table}

Table~\ref{tbl:k-means-euc-direct} pictures the crosstab between the GICS classification on the left and the clusters determined by the k-means algorithm at the top. A Chi-squared contingency test results in a $\chi^2$ value of $113.58$, which corresponds to a p-value of $0.1667$. Thus the Null hypothesis that the GICS classification and k-means cluster are stochastically independent of each other cannot be discarded on a 5~\% level of significance. This means that there is no significant correspondence between the GICS classes and the clusters found by k-means.

A possible cause for the lack of equivalence could be the way the distance is calculated. Equation~\ref{eq:dist-direct} uses the absolute stock prizes. That way the distance of two stocks with a similar trend can be bigger, due to large differences in the absolute stock prize, than between two stocks with different progression but close stock prizes.


